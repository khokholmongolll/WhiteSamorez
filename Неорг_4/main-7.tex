\documentclass[a4paper, 12pt]{article}
\usepackage[a4paper,top=1.5cm, bottom=1.5cm, left=1cm, right=1cm]{geometry}
\usepackage{cmap}					
\usepackage{mathtext} 				
\usepackage[T2A]{fontenc}			
\usepackage[utf8]{inputenc}			
\usepackage[english,russian]{babel}
\usepackage{multirow}
\usepackage{graphicx}
\usepackage{wrapfig}
\usepackage{tabularx}
\usepackage{float}
\usepackage{longtable}
\usepackage{hyperref}
\hypersetup{colorlinks=true,urlcolor=blue}
\usepackage[rgb]{xcolor}
\usepackage{amsmath,amsfonts,amssymb,amsthm,mathtools} 
\usepackage{icomma} 
\usepackage{euscript}
\usepackage{mathrsfs}
\usepackage{enumerate}
\usepackage{caption}
\usepackage{enumerate}
\usepackage{graphicx}
\usepackage{caption}
\usepackage{subcaption}
\usepackage{wasysym }
\usepackage{ifthen}
\usepackage{calc}
\newcommand{\RomanNumeralCaps}[1]
    {\MakeUppercase{\romannumeral #1}}
\title{\textbf{Лабораторная работа по неорганической химии №4. Окислительно-восстановительные реакции.}}
\author{Коротков Антон и Хохлов Андрей Б06-302}
\date{9 марта 2024}




\begin{document}

	
	\maketitle
	
	\section{Реакции с участием кислорода воздуха}
 \subsection{Реакции гидроксидов металлов в промежуточной степени окисления с кислородом воздуха.}
 Соль Мора при взаимодействии с едким натром будет образовывать гидроксид железа (\RomanNumeralCaps{2}) и аммиак:
 \begin{equation}
     FeSO_4 \cdot (NH_4)_2SO_4\cdot 6H_2O + 4NaOH \xrightarrow{} Fe(OH)_2 \downarrow + 2NH_3 + 10H_2O + 2Na_2SO_4
 \end{equation}
 Как известно, чистый гидроксид железа (\RomanNumeralCaps{2}) - белый, но ничего даже близкого к белому осадку мы не наблюдаем. У нас в пробирке зеленовато-бурый осадок, который, с течением времени становится все более и более бурым. Это очень просто обьясняется тем, что гидроксид железа (\RomanNumeralCaps{2}) хорошо окислияется на воздухе и переходит в бурый гидроксид железа (\RomanNumeralCaps{3}):
 \begin{equation}
     4Fe(OH)_2 + O_2 + 2H_2O \xrightarrow{} 4Fe(OH)_3 \downarrow
 \end{equation}
 Аналогичное явление наблюдается и с марганцем, только у него соединения +3 не отличаются стабильностью, и образовываться будет бурый осадок $MnO_2 \cdot 2H_2O$ :
 \begin{equation}
     MnSO_4 + 2NaOH \xrightarrow{} Mn(OH)_2 \downarrow + Na_2SO_4
 \end{equation}
 \begin{equation}
     2Mn(OH)_2 + O_2 + 2H_2O \xrightarrow{} 2MnO_2 \cdot 2H_2O \downarrow
 \end{equation}
 \subsection{Горение металлов на воздухе}
 Подожжем магниевую стружку с помощью спиртовки:
 \begin{equation}
     2Mg + O_2 \xrightarrow[]{} 2MgO
 \end{equation}
К сожалению, у нас не было с собой солнцезащитных очков, поэтому во время опыта в целях безопасности пришлось отвернуться. По данной причине описать наблюдения не получится. \newline
Полученное вещество - жженая магнезия, а жженая магнезия, как известно, с водой не реагирует (при нормально температуре, а реагирует легкая магнезия)
\section{Окислительные свойства ионов металлов в высоких степенях окисления}
Как известно, ион железа 3+ является довольно неплохим окислителем. В нашем случае он будет окислять ион олова 2+:
\begin{equation}
    Fe^{3+} + Sn^{2+} \xrightarrow[]{} Fe^{2+} + Sn^{4+}
\end{equation}
 Изначально после добавления роданида натрия к раствору хлорида железа(\RomanNumeralCaps{3}) наблюдается окрашивание раствора в интенсивно-красный цвет:
\begin{equation}
    FeCl_3 + NaSCN +5H_2O \xrightarrow[]{} [Fe(H_2O)_5(SCN)]Cl_3
\end{equation}
Затем после добавления хлорида олова можно наблюдать частично обесцвечивание и пожелтение раствора в следствие ОВР между оловом 2+ и железом 3+:
\begin{equation}
    FeCl_3 + SnCl_2 \xrightarrow[]{} FeCl_2 + SnCl_4
\end{equation}
\section{Окислительная способность p-элементов в высших степенях окисления}
 Как известно, азотная, серная, хлорноватая - сильные кислоты, а значит их натриевые соли подвергаться гидролизу не будут. А вот угольная и фосфорная - кислоты слабые, и их средние натриевые соли имеют щелочную среду. Опыт это подтверждает, в пробирках с фосфатом и карбонатом натрия индикаторная бумажка посинела, в то время как в других пробирках ее цвет не изменился.
 Вообще говоря, данные нам соли в растворах особо не проявляют окислительных свойств (в особенности это касается фосфата и карбоната натрия), например в случае фосфата это обусловлено большим радиусом фосфора, отсутствием кайносимметричности и прочными одинарными связями. Опыт подтверждает такое мнение. Ни в одной из пробирок после подкисления и добавления йодида калия, к сожалению, не произошла реакция.
 \section{Окислительные и восстановительные свойства пероксида водорода.}
Между йодидом калия и пероксидом водорода в кислой среде идет ОВР:
\begin{equation}
    2KI + H_2O_2 + H_2SO_4 \xrightarrow[]{} I_2 + K_2SO_4 + 2H_2O
\end{equation}
Появление коричневой окраски в данной реакции обусловлено выделением йода. Восстановителем является йодид калия($I^{-1}$), окислителем - пероксид водорода ($O^{-1}$). \newline
Между дихроматом калия и пероксидом водорода в кислой среде также будет протекать окислительно-восстановительная реакция (пероксид водорода может проявлять как окислительные, так и восстановительные свойства):
\begin{equation}
    3H_2O_2 + K_2Cr_2O_7 + 4H_2SO_4 \xrightarrow[]{} 3O_2 + K_2SO_4 + Cr_2(SO_4)_3 + 7H_2O
\end{equation}
В ходе данной реакции наблюдается выделение бесцветного газа(кислорода) и изменение цвета раствора с оранжевого на голубой.
\section{Факторы, влияющие на протекание ОВР}
\subsection{Влияние pH}
Реакции, протекающие между перманганатом и сульфитом в разных средах:
\begin{equation}
     5Na_2SO_3 + 2KMnO_4 + 3H_2SO_4 \xrightarrow[]{} 5Na_2SO_4 + K_2SO_4 + 2MnSO_4 + 3H_2O 
    \end{equation}
    \begin{equation}
    3Na_2SO_3 + 2KMnO_4 + H_2O \xrightarrow[]{} 3Na_2SO_4 + 2MnO_2 \downarrow + 2KOH
    \end{equation}
    \begin{equation}
    Na_2SO_3 + 2KMnO_4 + 2NaOH \xrightarrow[]{} Na_2SO_4 + K_2MnO_4 + Na_2MnO_4 + H_2O
    \end{equation}
В кислой среде наблюдается обесцвечивание раствора, в нейтральной - обесцвечивание и выпадение бурого осадка, в щелчной - изменение цвета раствора с фиолетового на зеленый.
Потенциалы полуреакций:
\begin{table}[H]
    \centering
    \begin{tabular}{|c|c|c|}\hline
       среда & $KMnO_4$ & $Na_2SO_3$ \\  \hline
        кислая &1.51 &	0.172 \\ \hline
        нейтральная & 0.57 & -0.22 \\ \hline
        щелочная &0.564 & -0.93 \\ \hline
    \end{tabular}
    \caption{Таблица потенциалов (В)}
    \label{tab:my_label}
\end{table}
\subsection{Влияние концентрации}
Между растворами 0.1 М дихромата калия и 2М соляной кислоты окислительно-восстановительная реакция на протекает. Но, если взять твердый дихромат, то реакция пойдет:
\begin{equation}
    K_2Cr_2O_7 + 14HCl \xrightarrow[]{} 2KCl + 2CrCl_3 + 3Cl_2 + 7H_2 O
\end{equation}
Наблюдается выделение бесцветного газа и окрашивание раствора в зеленый цвет.
\subsection{Влияние температуры}
Температура ускоряет химические реакции, и, в частности овр. В нашем случае при нагревании первой пробирки (с двумя растворами) реакция так и не пошла, а в случае нагревания второй пробирки (куда добавляли твердый дихромат) реакция пошла интенсивнее.
\subsection{Влияние природы вещества.}
При добавлении к перманганату калия концентрированной серной кислоты произойдет образование оксида марганца:
\begin{equation}
    2KMnO_4 + H_2SO_{4 ~\text{конц.}} \xrightarrow[]{} Mn_2O_7
 + K_2SO_4 + H_2O\end{equation}
При дальнейшем добавлении спирта он будет окислен оксидом марганца:
\begin{equation}
    6Mn_2O_7 + 5C_2H_5OH + 12H_2SO_4 \xrightarrow[]{} 6MnSO_4 + 10CO_2 + 27H_2O
\end{equation}
\section{Взаимодействие металлов с концентрированной и разбавленной азотоной кислотой}
\subsection{Разбавленная азотная кислота}
Как известно, разбавленная азотная кислота в реакциях с металлами способна давать довольно широкий спектр продуктов: начиная от нитрата аммония и заканчивая монооксидом азота. В наших реакциях мы наблюдали:
\begin{equation}
    5Mg + 12HNO_3 \xrightarrow[]{} 5Mg(NO_3)_2 + N_2\uparrow + 6H_2O
\end{equation}
\begin{equation}
    Fe + 4HNO_3 \xrightarrow[]{} Fe(NO_3)_3 + NO\uparrow + 2H_2O
\end{equation}
\begin{equation}
    Cu + HNO_3 \xrightarrow[]{} \text{реакция не идет}
\end{equation}
Как известно, чем более активный металл, тем более он хороший восстановитель. Поэтому, чем более активный металл, тем сильнее должна восстанавливаться азотная кислота. Это мы и наблюдаем на опыте: в случае с магнием заметно выделение газа без цвета и запаха, индикаторная бумажка не реагирует на него, он не буреет на воздухе $\Rightarrow$ это азот. В случае с железом также выделяется бесцветный газ, но он уже буреет на воздухе $\Rightarrow$ это монооксид азота. А вот с медью реакция не пошла, судя по всему она спассивировала.
\subsection{Концентрированная азотная кислота}
Концентрированная "азотка" не отличается таким разнообразием, и чаще всего в ее реакциях с металлами "травит" бурый газ, также известный как диоксид азота. В наших реакциях мы наблюдали:
\begin{equation}
    Mg + 4HNO_3 \xrightarrow[]{} Mg(NO_3)_2 + 2NO_2 \uparrow + 2H_2O
\end{equation}
\begin{equation}
    Fe + 6HNO_3 \xrightarrow[]{} Fe(NO_3)_3 + 3NO_2\uparrow  + 3H_2O
\end{equation}
\begin{equation}
    Cu + 4HNO_3 \xrightarrow[]{} Cu(NO_3)_2 + 2NO_2 \uparrow + 2H_2 O
\end{equation}
Во всех реакциях выделился бурый газ. \newline
В силу своей окислительной способности азотная кислота спосбна реагировать не только с металлами, но и со сложными веществами, например с йодидом калия:
\begin{equation}
    2KI + 4HNO_3 \xrightarrow[]{} 2KNO_3 + 2NO_2\uparrow + I_2 + 2H_2O 
\end{equation}
В ходе данной реакции наблюдается выделение бурого газа и окрашивание раствора в бурый цвет (за счет образования йода)
\section{Взаимодействие концентрированной и разбавленной серной кислоты с металлами разной активности}
\subsection{Концентрированная серная кислота}
Как известно, концентрированная серная кислота является кислотой окислителем. Чаще всего в реакциях с металлами она дает сернистый газ или сероводород. В редких случаях возможно образование серы. В наших реакциях мы наблюдали:
\begin{equation}
    4Mg + 5H_2SO_4 \xrightarrow[]{} 4MgSO_4 + H_2S\uparrow + 4H_2O
\end{equation}
\begin{equation}
    2Fe + 6H_2SO_4 \xrightarrow[]{} Fe_2(SO_4)_3 + 3SO_2 \uparrow+ 6H_2O
\end{equation}
    \begin{equation}
        2Cu + H_2SO_4 \xrightarrow[]{} CuSO_4 + SO_2 \uparrow + 2H_2O
    \end{equation}
В реакции с магнием явно образовалась смесь продуктов, но, судя по характерному запаху тухлых яиц превалирующим продуктом был сероводород, что вполне возможно, тк магний активный металл. С железом и медью однозначно наблюдалось выделение диоксида серы.
\subsection{Разбавленная серная кислота}
Разбавленная серная кислота, как известно, является обычной кислотой (неокислителем), и , как следствие реагирует только с металлами, стоящими в ряду активности до водорода. В наших реакциях мы наблюдали:
\begin{equation}
    Mg + H_2SO_4 \xrightarrow[]{} MgSO_4 + H_2 \uparrow
\end{equation}
\begin{equation}
    Fe + H_2SO_4 \xrightarrow[]{} FeSO_4 + H_2 \uparrow
\end{equation}
\begin{equation}
    Cu + H_2SO_4 \xrightarrow[]{} \text{реакция не идет}
\end{equation}
С магнием и железом наблюдалось выделение водорода. а с медью по очевидным причинам реакция не идет.
\section{Химические часы(реакция Ландольта)}
Взаимодействие $Na_2SO_3$ с $KIO_3$ в кислой среде. Продукт — $I_2$ — диагностируем раствором крахмала.
В результате реакции бесцветные реагенты переходят в окрашенные продукты.
Мы определяем скорость реакции по времени появления окраски.
Если мы уменьшаем концентрацию иодата в 2 раза,
то время реакции увеличивается в 2 раза, т.е. скорость реакции уменьшается в 2 раза.
Если сравнить время протекания реакции с начальной концентрацией C, C/2 и C/5, то можно установить зависимость скорости реакции от концентрации иодата, т.е. порядок реакции по IO3-. У нас идет реакция первого порядка:
\begin{equation}
  IO_3^{-} + 5I^{-} +6H^{+} \xrightarrow[]{}  3I_2 + 3H_2O 
\end{equation}
Лимитирующая стадия:
\begin{equation}
  IO_3^{-} + 3SO_4^{2-} \xrightarrow[]{} I^{-} +SO_4^{2-} 
\end{equation}
Скорость зависит от концентрации линейно, потому что концентрация иодата в кинетическом уравнении лимитирующей стадии — в первой степени.
\section{Вывод}
\begin{itemize}
    \item Повторили основных окислителей и востановителей, на практике проверили влияние различных условий на протекание ОВР.
\end{itemize}
\end{document}
