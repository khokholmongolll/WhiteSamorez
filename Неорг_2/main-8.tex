\documentclass[a4paper, 12pt]{article}
\usepackage[a4paper,top=1.5cm, bottom=1.5cm, left=1cm, right=1cm]{geometry}
\usepackage{cmap}					
\usepackage{mathtext} 				
\usepackage[T2A]{fontenc}			
\usepackage[utf8]{inputenc}			
\usepackage[english,russian]{babel}
\usepackage{multirow}
\usepackage{graphicx}
\usepackage{wrapfig}
\usepackage{tabularx}
\usepackage{float}
\usepackage{longtable}
\usepackage{hyperref}
\hypersetup{colorlinks=true,urlcolor=blue}
\usepackage[rgb]{xcolor}
\usepackage{amsmath,amsfonts,amssymb,amsthm,mathtools} 
\usepackage{icomma} 
\usepackage{euscript}
\usepackage{mathrsfs}
\usepackage{enumerate}
\usepackage{caption}
\usepackage{enumerate}
\usepackage{graphicx}
\usepackage{caption}
\usepackage{subcaption}
\usepackage{wasysym }
\usepackage{ifthen}
\usepackage{calc}
\newcommand{\RomanNumeralCaps}[1]
    {\MakeUppercase{\romannumeral #1}}
\title{\textbf{Лабораторная работа №2. Приготовление растворов, кислотно-основное титрование.}}
\author{Коротков Антон и Хохлов Андрей Б06-302}
\date{24 февраля 2024}




\begin{document}

	
	\maketitle
	\section{Ход работы}
	\subsection{Приготовление 1М раствора гидроксида натрия.  }
Гидроксид натрия представляет собой твердое вещество белого цвета, которое при хранении на воздухе расплывается, так как поглощает влагу (т. е. гигроскопичен). Помимо гидратации гидроксид натрия постепенно превращается в карбонат за счет связывания углекислого газа из воздуха:
\begin{equation}
    2NaOH + CO_2 \xrightarrow{} Na_2CO_3 + H_2O
\end{equation}

Перед выполнением опыта проверяем выданный нам гидроксид натрия на наличие в нем примеси карбоната. Для этого одну крупинку твердого вещества растворяем в 1 мл воды и добавляем 5 процентный раствор хлорида бария. Образование осадка или появление мути свидетельствует о примеси карбоната:
\begin{equation}
    Na_2CO_3 + BaCl_2 \xrightarrow[]{} BaCO_3\downarrow + 2NaCl
\end{equation}

В нашем случае карбонат в щелочи отсутствует, значит ее можно использовать для приготовления раствора.
Рассчитаем массу гидроксида натрия, которая требуется для приготовления 100 мл 1М раствора.
\begin{align}
    C = \frac{\nu}{V},\quad \nu = \frac{m}{\mu} \Rightarrow m = C \cdot V \cdot \mu =1 \cdot 0.1 \cdot 40 = 4 \text{г}
\end{align}
Отвесим необходимое количество щелочи в (пластиковом) стаканчике, приблизительно равное рассчитанному. Взвешиваем быстро, т.к. вследствие гигроскопичности щелочи ее навеска быстро увеличивает вес. Кусочки твердой щелочи достаточно крупные, поэтому масса навески не должна точно совпадать с расчетной. В данной работе допустимы отклонения до 20 процентов. Однако запишем точную массу отвешенной щелочи: $m = 3,96$ г. Растворим навеску в небольшом объеме дистиллированной воды и перенесем его через воронку в мерную колбу объемом 100 мл. Тщательно смоем остатки щелочи со стенок стаканчика дистиллятом и сольем этот раствор в ту же мерную колбу, после чего ополоснем воронку и вынем ее из горлышка колбы. Заполним колбу водой, все время перемешивая раствор плавными движениями руки. Это легко делать пока колба не полностью заполнена водой. Доведем объем раствора точно до метки так, чтобы нижний край мениска (когда он находится на уровне глаз) находился напротив метки.
\subsection{Определение концентрации раствора едкого натра методом  кислотно-основного титрования}
Разбавим полученный ранее раствор гидроксида натрия в 10 раз так, чтобы получилось 250 мл разбавленного раствора щелочи. Для этого с помощью пипетки отберем 25 мл концентрированного раствора NaOH и перенесем в мерную колбу на 250 мл. Заполним колбу примерно на половину дистиллятом и перемешаем содержимое легким вращением колбы. Затем заполним колбу дистиллятом точно до метки так, чтобы против нее находился нижний край мениска (когда он находится на уровне глаз). Колбу с раствором плотно закроем крышкой и тщательно перемешаем переворачиванием, крепко придерживая крышку пальцем.\newline
Вставим воронку в бюретку и заполним её разбавленным раствором гидроксида натрия. Сольем эту первую порцию раствора в отдельный стаканчик, открыв кран бюретки для ее промывки. Слив  прекратим, когда уровень жидкости будет находиться чуть выше крана. Снова заполним бюретку, наливая раствор немного выше деления, принятого за начало отсчёта.  После этого вновь нальем раствор выше нулевой метки. Вынем воронку из верхней части бюретки и установим исходный нулевой уровень раствора, сливая избыток щелочи в стаканчик.
В коническую колбу при помощи мерной пипетки отберем аликвоту ранее приготовленного стандартного раствора 0,1М HCl объемом 10 мл и добавим две капли раствора фенолфталеина, служащего индикатором. Далее проведем титрование до момента нейтрализации кислоты щелочью, добавляя по каплям раствор щелочи из бюретки и непрерывно перемешивая жидкость в конической колбе плавными вращательным движениями руки:
\begin{equation}
    NaOH + HCl \xrightarrow[]{} NaCl + H_2O
\end{equation}Момент нейтрализации устанавливаем по появлению неисчезающей слабо-розовой окраски фенолфталеина (устойчивой в течение 30 с) от последней добавленной капли раствора щелочи. После окончания титрования определяем показания бюретки по нижнему краю мениска жидкости. Повторим титрование еще два раза. Результаты титрования занесем в таблицу:
\begin{table}[H]
    \centering
    \begin{tabular}{|c|c|c|c|c|} \hline
      \textnumero\quadопыта  & $V_{NaOH}$, мл   &  $V_{HCl}$ (0.1н), мл   & $C_{NaOH}, \frac{\text{моль}}{\text{л}}$    &$ \langle C_{NaOH} \rangle $ \\ \hline
        1   &  10.2  &   10  &  0.098  &    \\ \hline  
         2 &  10.2  &   10  &   0.098 & 0.098   \\ \hline
        3    & 10.2   &   10  &  0.098  &    \\ \hline
    \end{tabular}
    \caption{Результаты титрования}
    \label{tab:my_label}
    Значение, вычисленное по массе навески:
    \begin{equation}
        C = \frac{\nu}{V} = \frac{m}{\mu \cdot V} = 0.099 \frac{\text{моль}}{\text{л}}
    \end{equation}
\end{table}
Таким образом, погрешность крайне мала. Наличие погрешности может быть связано с тем, что щелочь "надышалась" \quad$CO_2$.
\subsection{Определение массовой доли примеси глюкозы, содержащейся в щавелевой кислоте}
После точного определения молярной концентрации раствора гидроксида натрия методом кислотно-основного титрования (стандартизации) его можно использовать в качестве рабочего раствора для количественного анализа кислот. При количественном анализе концентрированной кислоты ее предварительно разбавляют так, чтобы объем титранта не превышал объем бюретки. В случае, когда в кислоте имеется примесь, не влияющая на pH раствора и не взаимодействующая с титрантом, можно определить количество этой примеси, оттитровав кислоту.
Возьмем навеску в количестве 1,5 г щавелевой кислоты, содержащую кристаллизационную воду (H2C2O4·2H2O) и примесь глюкозы. Приготовим из этой навески рабочий раствор в мерной колбе на 100 мл. Пипеткой отберем 10 мл аликвоты полученного раствора и перенесем его в коническую колбу ёмкостью 100 мл, добавим две капли раствора фенолфталеина. Бюретку снова заполним раствором гидроксида натрия и оттитруем раствор щавелевой кислоты до появления бледно-розовой окраски, устойчивой в течение 30 с. 
\begin{equation}
    2NaOH + HOOC-COOH \xrightarrow[]{} NaOOC-COONa + 2H_2O
\end{equation}
Результаты титрования:
\begin{table}[H]
    \centering
    \begin{tabular}{|c|c|c|c|c|} \hline
      \textnumero\quadопыта  & $V_{\text{Аликвоты}}$, мл   &  $V_{NaOH}$, мл   & $C_{H_2C_2O_4}, \frac{\text{моль}}{\text{л}}$    &$ \langle C_{H_2C_2O_4} \rangle $ \\ \hline
        1   &  10  &   2  &  0.0098  &    \\ \hline  
         2 &  10  &   2  &   0.0098 & 0,01029   \\ \hline
        3    & 10   &   2,3  &  0.01127  &    \\ \hline
    \end{tabular}
    \caption{Результаты титрования}
    \label{tab:my_label}
    \end{table}
    
\begin{align}
     & C_{H_2C_2O_4} = \frac{C_{NaOH} \cdot V_{NaOH}}{ V_{H_2C_2O_4} } \Rightarrow \langle C_{H_2C_2O_4} \rangle \approx 0,01029 \quad \text{М} \\ \newline
    & \langle \nu_{H_2C_2O_4} \rangle = \langle C_{H_2C_2O_4} \rangle \cdot V_{H_2C_2O_4} \approx 0,0001029 \quad \text{моль} \\
    & \nu_{\text{теор.}} = \frac{m_{H_2C_2O_4}}{\mu_{H_2C_2O_4} \cdot 100} \approx   0,00012 \quad \text{моль}   \\
    & \omega_{\text{прим.}} = 1- \frac{m_{\text{практич.}}}{m_{\text{теор.}}} = 1 - \frac{\nu_{\text{практ.}}}{\nu_{\text{теор.}}} \approx 0,1425 \approx 14,3 \%
\end{align}
Таким образом, массовая доля глюкозы в смеси 14,3 \%.

\section{Вывод.}
Мы научились готовить растворы и титровать, а также определять количество примесей при помощи кислотно-основного титрования.
\end{document}

Результаты титрования занесите в табл. 2.4, вычислите молярную концентрацию щавелевой кислоты в приготовленном растворе, учитывая, что в реакции нейтрализации участвую обе кис-лотные группы, а сама она находится в виде дигидрата H2C2O4⋅2H2O.
Определите массовую долю глюкозы в смеси, исходя из массы навески
