\documentclass[a4paper,12pt]{article} 
\usepackage{geometry}
\geometry{
	a4paper,
	total={170mm,257mm},
	left=20mm,
	top=20mm,
}
\usepackage{titlesec}
\titlelabel{\thetitle.\quad} %точка в section

%%% Работа с русским языком
\usepackage{cmap}                           % поиск в PDF
\usepackage{mathtext} 			 	       % русские буквы в формулах
\usepackage[T2A]{fontenc}               % кодировка
\usepackage[utf8]{inputenc}              % кодировка исходного текста
\usepackage[english,russian]{babel}  % локализация и переносы

%Математика
\usepackage{amsmath,amsfonts,amssymb,amsthm,mathtools} % AMS
\usepackage{icomma} % "Умная" запятая

%% Шрифты
\usepackage{euscript}	 % Шрифт Евклид
\usepackage{mathrsfs} % Красивый матшрифт

%% Команды
\DeclareMathOperator{\const}{\mathop{const}}

%% Перенос знаков в формулах
%\newcommand*{\hm}[1]{#1\nobreak\discretionary{}
%	{\hbox{$\mathsurround=0pt #1$}}{}}
\usepackage[pdftex]{graphicx}

%%% Заголовок
\author{Хохлов Андрей, Коротков Антон}
\title{Практическая работа 9 \\
	\textbf{Свойства переходных металлов и их соединений}}
\date{28 апреля 2024}

\begin{document}
{\Large \maketitle}
\section{Кислотно-основные свойства соединений хрома(III)}
При добавлении к хлориду хрома три недостатка щелочи выпадает серо-зелёный осадок
\begin{equation} 
\mathrm{CrCl_3 + 3NaOH \longrightarrow 3NaCl + Cr(OH)_3 \downarrow } 
\end{equation}
Разделили осадок на 2 пробирки, в первой пробирке:
\begin{equation} 
\mathrm{Cr(OH)_3 + 3NaOH \longrightarrow Na_3[Cr(OH)_6]  } 
\end{equation}
Во второй пробирке:
\begin{equation} 
\mathrm{2Cr(OH)_3 + 6HCl \longrightarrow  2CrCl_3 + 3H_2O}
\end{equation}
Альтернативный опыт
\begin{equation} 
2CrCl_3 + 3Na_2 CO_3 + 3H_2O \longrightarrow 2Cr(OH)_3 + 6NaCl + 3CO_2
\end{equation}
Сульфид хрома гидролизуется необратимо:
\begin{equation} 
\mathrm{2CrCl_3 + 3Na_2S + 6H_2O \longrightarrow 3H_2S + 2Cr(OH)_3 + 6NaCl }
\end{equation}
\section{Окислительно-восстановительные свойства соединений хрома(III)} 
Соединения хрома три способны проявлять окислительные свойства:
\begin{equation}
    CrCl_3 \xrightarrow[- ZnCl_2]{+Zn + HCl} CrCl_2
\end{equation}
В щелочной среде соединения хрома три способны проявлять восстановительные свойства:
\begin{equation}
    Cr(OH)_3 + NaOH + H_2O_2 \longrightarrow Na_2CrO_4 + H_2O
\end{equation}
\begin{equation}
    Cr(OH)_3 + NaOH + NaClO \longrightarrow Na_2CrO_4 + NaCl + H_2O
\end{equation}
\section{Равновесие «хромат-дихромат» и его зависимость от кислотности среды}
При добавлении кислоты хромат переходит в дихромат:
\begin{equation} 
\mathrm{2Na_2CrO_4 + H_2SO_4 \longrightarrow Na_2Cr_2O_7 + Na_2SO_4+H_2O }
\end{equation}
Добавление щёлочи возвращает дихромат к хромату
\begin{equation} 
\mathrm{Na_2Cr_2O_7 + 2NaOH \longrightarrow 2Na_2CrO_4 +H_2O }
\end{equation}
Затем, разделили полученный дихромат по трём пробиркам, в первой пробирке добавили недостаток хлорида бария и реакция не пошла, с ацетатом реакция тоже не пошла, осадка не было, а вот в избытке хлорида бария выпал белый осадок
\begin{equation} 
\mathrm{Na_2Cr_2O_7 + BaCl_2 \longrightarrow BaCrO_4 + CrO_3 + 2NaCl}
\end{equation}
При разбавлении, дихромат начинал приобретать жёлтую окраску, значит при уменьшении концентрации равновесие сдвигается в сторону хромата.
\section{Окислительные свойства дихромата калия}
Рассмотрим реакцию с хлороводородом, в растворе реакция не пошла, а вот когда дихромат был сухой, пошла реакция с выделением хлора.
\begin{equation} 
\mathrm{Na_2Cr_2O_7 + 14HCl \longrightarrow 3Cl_2 + 2CrCl_3 + 2NaCl + 7H_2O }
\end{equation}
C  сульфидом натрия тоже пошла реакция:
\begin{equation} 
\mathrm{Na_2Cr_2O_7 + 3Na_2S + 7H_2O \longrightarrow 2S + 3Cr(OH)_3 + 8NaOH }
\end{equation}
\section{Окислительно-восстановительные свойства марганца и ванадия в высших
степенях окисления}
При добавлении глюкозы к перманганату в кислой среде перманганат обесцветился, уравнивание этой реакции отсавим читателю в качестве \textbf{небольшого упражнения}
\begin{equation} 
\mathrm{KMnO_4 + H_2SO_4 + Glu \longrightarrow CO_2 + MnSO_4 + K_2SO_4 + H_2O }
\end{equation}
В нейтральной среде реакция не пошла, а в щелочной среде выпал зелёный осадок манганата калия, переходящий в бурый.
\begin{equation} 
\mathrm{KMnO_4 + NaOH + Glu \longrightarrow Glu-Acid-Na + K_2MnO_4 + H_2O }
\end{equation}
При добавлении хлорида бария выпадает осадок перманганта бария
\begin{equation} 
\mathrm{2KMnO_4 + BaCl_2 \longrightarrow Ba(MnO_4)_2 + 2KCl }
\end{equation}
C серной кислотой будет переосаждаться до сульфата бария:
\begin{equation} 
\mathrm{Ba(MnO_4)_2  + H_2SO_4 \longrightarrow BaSO_4 + HMnO_4 }
\end{equation}
Потом, мы начали делать Ванадиевую радугу, для этого раствор ортованадата натрия подкисляют, а затем вносят металлический цинк, выделяющийся водород восстанавливает ванадий.
\begin{equation} 
\mathrm{VO_4^{-3} \longrightarrow VO_2^{2+}(синий) \longrightarrow V^{3+}(зелёный)\longrightarrow V^{2+}(фиолетовый)}
\end{equation}

\section{Разложение перманганата калия}
\begin{equation} 
\mathrm{2KMnO_4  \longrightarrow K_2MnO_4 + MnO_2 + O_2}
\end{equation}
Лучинка загорается из-за кислорода, в осадок выпадает манганат зелёный, потом оксид тёмно-коричневого цвета.
\section{Химические свойства железа}
По результатам эксперимента, у нас разбавленная пассивировала(кек), но всё равно данные реакции справедливы:
Для разбавленной:
\begin{equation} 
\mathrm{Fe + H_2SO_4  \longrightarrow FeSO_4 + H_2}
\end{equation}
Для концентрированной:
\begin{equation} 
\mathrm{Fe + H_2SO_4  \longrightarrow Fe_2(SO_4)_3 + SO_2 + H_2O}
\end{equation}
После внесения порошка железа он начал покрываться медью.
\begin{equation} 
\mathrm{Fe + CuSO4  \longrightarrow Cu + FeSO_4}
\end{equation}
\section{Окислительно-восстановительные свойства железа(II) и железа(III)}
\begin{equation} 
\mathrm{FeCl_3 + Na_2SO_3 + H_2O  \longrightarrow FeCl_2 + Na_2SO_4 + 2HCl}
\end{equation}
\begin{equation} 
\mathrm{FeCl_3 + 2KI \longrightarrow FeCl_2 +I_2 + 2KCl}
\end{equation}
\begin{equation} 
\mathrm{FeCl_3 + 2KI \longrightarrow FeCl_2 +I_2 + 2KCl}
\end{equation}
\begin{equation} 
\mathrm{FeSO_4 + 2KMnO_4 + 8H_2SO_4 \longrightarrow K_2SO_4 + MnSO_4 + 5Fe_2(SO_4)_3 + 8H_2O}
\end{equation}
\section{Качественные реакции на ионы железа(II) и железа(III)}
\begin{equation} 
\mathrm{[Fe(H_2O)_6]Cl_3 + NaSCN \longrightarrow [Fe(SCN)(H_2O)_5]Cl_2 + NaCl + H_2O}
\end{equation}
Получение берлинской лазури
\begin{equation} 
\mathrm{4FeCl_3 +3 K_4[Fe(CN)_6] \longrightarrow Fe_4[Fe(CN)_6]_3+12KCl }
\end{equation}
Получение турнбулевой сини
\begin{equation} 
\mathrm{3FeSO_4 + 2K_3[Fe(CN)_6] \longrightarrow Fe_3[Fe(CN)_6]_2+3K_2SO_4 }
\end{equation}
\section{Взаимодействие цинка с растворами кислот и щелочей}
\begin{equation} 
\mathrm{Zn + HCl  \longrightarrow ZnCl_2 + H_2}
\end{equation}
\begin{equation} 
Zn + NaOH + H_2O \longrightarrow Na_2[Zn(OH)_4] + H_2
\end{equation}
\section{Окисление аммиачного комплекса кобальта (II)}
Аммиачный комплекс кобальта розового цвета
\begin{equation} 
\mathrm{CoCl_2 + 6NH_3 \longrightarrow [Co(NH_3)_6]Cl_2}
\end{equation}
\section{Получение соли Шевреля и её аналогов}
До прокаливаня была получена соль Шевреля
\begin{equation} 
\mathrm{CuSO_4 + H_2SO_4 + Na_2SO_4 \longrightarrow Cu_2SO_3 \cdot CuSO_3 \cdot 2H_2O + SO_2}
\end{equation}
После прокаливания получили оксид меди (I)
\end{document}