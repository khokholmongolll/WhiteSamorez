 \documentclass[a4paper,12pt]{article}
\usepackage[a4paper,top=1.3cm,bottom=2cm,left=1.5cm,right=1.5cm,marginparwidth=0.75cm]{geometry}
\usepackage{setspace}
\usepackage{cmap}					
\usepackage{mathtext} 				
\usepackage[T2A]{fontenc}			
\usepackage[utf8]{inputenc}			
\usepackage[english,russian]{babel}
\usepackage{multirow}
\usepackage{graphicx}
\usepackage{wrapfig}
\usepackage{tabularx}
\usepackage{float}
\usepackage{longtable}
\usepackage{hyperref}
\hypersetup{colorlinks=true,urlcolor=blue}
\usepackage[rgb]{xcolor}
\usepackage{amsmath,amsfonts,amssymb,amsthm,mathtools} 
\usepackage{icomma} 
\usepackage{euscript}
\usepackage{mathrsfs}
\usepackage{booktabs}
\usepackage{amsmath}
\newcommand{\RomanNumeralCaps}[1]
    {\MakeUppercase{\romannumeral #1}}



\title{\textbf{Лабораторная работа №7. Химические свойства неметаллов 5 и 6 групп.}}
\author{Коротков Антон и Хохлов Андрей Б06-302 }
\date{4.04.2024}


\begin{document}
	
	\maketitle


\section{Практическая часть}
\subsection{Опыт №1. Осаждение сульфидов и их свойства.}
Как известно, у сероводородной кислоты довольно большое количество солей являются осадками, также с некоторыми катионами металлов идет полный необратимый гидролиз. В нашем опыте мы наблюдали:
\begin{equation}
    Zn^{2+} + Na_2S \xrightarrow[]{} ZnS \downarrow + 2Na^+
\end{equation}
\begin{equation}
    Cu^{2+} + Na_2S \xrightarrow{} CuS\downarrow + 2Na^+
\end{equation}
\begin{equation}
    Fe^{2+}+ Na_2S \xrightarrow{} FeS\downarrow + 2Na^+
\end{equation}
\begin{equation}
    Fe^{3+}_{\text{хол.}} + 2Na_2S \xrightarrow{} FeS\downarrow + S\downarrow + 4Na^+
\end{equation}
\begin{equation}
    2Fe^{3+}_{\text{комн.}} + 3Na_2S  + 6H_2O\xrightarrow{} 2Fe(OH)_3\downarrow + 3H_2S\uparrow + 6Na^+
\end{equation}
\begin{equation}
    Mn^{2+} + Na_2S \xrightarrow[]{} MnS\downarrow + 2Na^+
\end{equation}
Сульфид цинка - белый осадок, меди - черный, железа (\RomanNumeralCaps{2}) - черный, марганца - 'телесного' цвета. Важно также отметить, что в случае реакции солей железа (\RomanNumeralCaps{3}) с сульфид анионом идут конкурирующие процессы(ОВР и гидролиз), но в случае холодного раствора превалируюет ОВР, а при комнатной температуре - гидролиз. Далее мы попробовали растворить полученные сульфиды в воде, растворились все кроме меди, запишем реакцию например для сульфида цинка (все реакции, в общем, одинаковые - получаем назад соль и при этом выделяется $H_2S$:
\begin{equation}
  ZnS + 2HCl \xrightarrow[]{} ZnCl_2 + H_2S\uparrow  
\end{equation}
А вот соль меди не растворилась, это связано с ее ОЧЕНЬ низким ПР.
\subsection{Опыт №2. Восстановительные свойства сульфидов.}
Как известно, сера в степени окисления -2 является довольно неплохим восстановителем, мы проверили это на реакциях с галогенами:
\begin{equation}
    I_2 + Na_2S \xrightarrow[]{} 2NaI + S\downarrow
\end{equation}
\begin{equation}
    Br_2 + Na_2S \xrightarrow{} 2NaBr + S\downarrow
\end{equation}
Также в реакциях с галогенами может происходить образование соединений серы и в более высоких степенях окисления(сульфатов,например)
\subsection{Опыт №3. Получение серы и ее диспропорционирование в щелочах.}
Сначала получаем серу по следующей реакции:
\begin{equation}
    Na_2S_2O_3 + H_2 SO_4 \xrightarrow[]{} Na_2SO_4 + SO_2 + S\downarrow + H_2O
\end{equation}
Далее приливаем туда щелочь и наблюдаем диспропорционирование:
\begin{equation}
    6NaOH + 3S \xrightarrow[]{} Na_2SO_3 + 2Na_2S + 3H_2O
\end{equation}
Диспропорционирование - тип ОВР, в которых один и тот же элемент поднимает и опускает СО.
\subsection{Опыт №4. Свойства тиосульфатов.}
Тиосульфаты - соли тиосерной кислоты, в них сера как бы находится в двух разных степенях окисления. Тиосульфаты являются довольно реакционно способными:
\begin{equation}
    Na_2S_2O_3 + 2HCl \xrightarrow[]{} 2NaCl + SO_2\uparrow + S\downarrow + H_2O
\end{equation}
\begin{equation}
    Na_2S_2O_3 + 5CL_2 + 5H_2O \xrightarrow[]{} 2NaCl + H_2SO_4 + 8HCl
\end{equation}
\begin{equation}
     Na_2S_2O_3 + Br_2 + H_2O \xrightarrow[]{} 2NaBr + S\downarrow + H_2SO_4
\end{equation}
\begin{equation}
     2Na_2S_2O_3 + I_2 \xrightarrow[]{} Na_2S_4O_6 + 2NaI
\end{equation}
\begin{equation}
    2Na_2S_2O_3 + 2Fe^{3+} \xrightarrow[]{} Na_2S_4O_6 + 2Fe^{2+} + 2Na^+
\end{equation}
\begin{equation}
     2Na_2S_2O_3 + Cu^{2+} \xrightarrow[]{} Na_2S_4O_6 + Cu^{+} + 2Na^+
\end{equation}
В реакции с соляной кислотой наблюдаем образование взвеси сери и выделение диоксида серы, в реакциях с галогенами - обесцвечивание их растворов, в реакциях с железом три и медью два - изменение цвета раствора.
\subsection{Опыт №5. Взаимодействие серной концентрированной кислоты с органическими и неорганическими веществами.}
Как всем известно, серная концентрированная кислота способна 'обугливать' сахара,что она собственно  и делает с сахарозой и целлюлозой(бумага). В случае с сахарозой наблюдали красивый 'подьем' обугленной массы. А вот опыт с углем, к сожалению, не пошел.
\subsection{Опыт №6. Свойства аммиака.}
Аммиак, как правило, изучается с двух сторон. Во-первых, как основание (неподеленная электронная пара на азоте делает свое дело), во-вторых - как восстановитель. В пункте А) данного опыта фенолфталеин стал малиновым:
\begin{equation}
    NH_3\cdot H_2O \xrightarrow[]{} NH_4^+ + OH^-
\end{equation}
В опыте б) нам предстоит подтвердить, что аммиак - более сильное основание чем все нерастворимые гидроксиды кроме гидроксида магния. В частности, он способен вытеснить железо (\RomanNumeralCaps{3}) из раствора его соли:
\begin{equation}
    Fe^{3+} + 3NH_3 + 3H_2O \xrightarrow[]{} Fe(OH)_3\downarrow + 3NH_4^{+}
\end{equation}
А в подпункте в) мы уже рассматриваем аммиак как восстановитель:
\begin{equation}
    2KMnO_4 + 2NH_3 \xrightarrow{} 2MnO_2\downarrow + N_2\uparrow + 2KOH + 2H_2O
\end{equation}
\subsection{Опыт №7. Свойства солей аммония. }
Между твердой солью аммония и гидроксидом кальция наблюдается следующая реакция:
\begin{equation}
    Ca(OH)_2 + 2NH_4Cl \xrightarrow[]{t^{\circ}} 2NH_3\uparrow + CaCl_2 + 2H_2O
\end{equation}
Выделяется аммиак, окрашивающие фенолфталеиновую бумажку в малиновый цвет.\newline
Карбонат аммония необратимо разлагается:
\begin{equation}
    (NH_4)_2CO_3 \xrightarrow[]{t^{\circ}} 2NH_3 + H_2O + CO_2
\end{equation}
 А хлорид аммония разлагается обратимо (хлороводород и аммиак не успевают улететь, реагируют, и хлорид аммония осжадается на стенках:
 \begin{equation}
     NH_4Cl \longleftrightarrow NH_3 \uparrow + HCl\uparrow
 \end{equation}
 \subsection{Опыт №8. Разложение нитрата калия.}
 При нагревании нитрата калия образуется расплав, состав которого довольно не понятен, но при разложении одназначно выделяется кислород, окисляюший уголь:
 \begin{equation}
     2KNO_3 \xrightarrow[]{t^{\circ}} 2KNO_2 + O_2
 \end{equation}
 \begin{equation}
     C + O_2 \xrightarrow[]{} CO_2
 \end{equation}
 Также в расплаве после разложения однозначно появляются восстановители, тк перманганат обесцвечивается:
 \begin{equation}
     2KMnO_4 + 5KNO_2 + 3H_2SSO_4 \xrightarrow[]{} 5KNO_3 + 2MnSO_4 + K_2SO_4 + 3H_2O
 \end{equation}
 \subsection{Опыт №9. Общие качественные реакции на анионы.}
 Как известно, свинец образует довольно много осадков, например мы получили:
 \begin{equation}
    Pb(NO_3)_2 + 2HCl \xrightarrow[]{} PbCl_2 \downarrow + 2HNO_3
 \end{equation}
 Выпал белый осадок.
  \begin{equation}
     Pb(NO_3)_2 + 2HBr \xrightarrow[]{} PbBr_2 \downarrow+ 2HNO_3
 \end{equation}
 Выпал желтоватый осадок.
  \begin{equation}
     Pb(NO_3)_2 + 2HI \xrightarrow[]{} PbI_2\downarrow + 2HNO_3
 \end{equation}
 Выпал желтый осадок.
  \begin{equation}
    Pb(NO_3)_2 + H_2O + CO_2 \xrightarrow[]{} PbCO_3\downarrow + 2HNO_3
 \end{equation}
 Выпал белый осадок.
  \begin{equation}
     Pb(NO_3)_2 + H_2SO_3 \xrightarrow[]{} PbSO_3 \downarrow + 2HNO_3
 \end{equation}
 Выпал белый осадок.
  \begin{equation}
   Pb(NO_3)_2 + H_2SO_4 \xrightarrow[]{} PbSO_4 \downarrow + 2HNO_3 
 \end{equation}
 Выпал белый осадок.
  \begin{equation}
  3Pb(NO_3)_2 + 2H_3PO_4 \xrightarrow[]{} Pb_3(PO_4)_2\downarrow + 6HNO_3   
 \end{equation}
 Выпал белый осадок. Приведем сокращенное ионное уравнение для реакции (27):
  \begin{equation}
     Pb^{2+} + 2Cl^- \xrightarrow[]{} PbCl_2\downarrow
 \end{equation}
И для реакции (31):
\begin{equation}
    Pb^{2+} + H_2SO_3 \xrightarrow[]{} PbSO_3\downarrow + 2H^+
\end{equation}
Для остальных реакций приводить сокрашенные РИО не имеет особого смысла, тк отличаться они будут только анионом.\newline
Все полученные осадки растворятся в концентрированной азотной кислоте, приведем уравнение для йодида и карбоната:
\begin{equation}
    PbI_2 + 4HNO_3 \xrightarrow[]{} Pb(NO_3)_2 + I_2 + 4NO_2 + 4H_2O
\end{equation}
\begin{equation}
    PbCO_3 + HNO_3 \xrightarrow[]{} Pb(NO_3)_2 + CO_2 + H_2O
\end{equation}
\subsection{Опыт №10. Обнаружение соединений серы.}
Сульфид ионы легко обнаружить по запаху тухлых яиц от взаимодействия с кислотами и по характерному почернению свинцовой бумажки:
\begin{equation}
    Na_2S + 2HCl \xrightarrow[]{} 2NaCl + H_2S\uparrow
\end{equation}
\begin{equation}
    Pb(NO_3)_2 + H_2S \xrightarrow[]{} PbS\downarrow + 2HNO_3
\end{equation}
Сульфат ион можно прекрасно обнаружить с помощью солей бария, причем необходима кислая среда, чтобы отличить именно сульфат от дургих внешне не отличающихся осадков бария:
\begin{equation}
  BaCl_2 + Na_2SO_4 \xrightarrow[]{H^+} BaSO_4\downarrow + 2NaCl 
\end{equation}
Сульфит ионы мы будем отличать, пользуясь их восстановительными свойствами, практически полным отсутствием окислительных и опять таки нерастворимостью сульфата бария:
\begin{equation}
    5Na_2SO_3 + 2KMnO_4 + 6HCl \xrightarrow[]{} 5Na_2SO_4 +  2KCl + 2MnCL_2 + 3H_2O
\end{equation}
После добавления сульфата бария идет реакция (39).
При добавлении к тиосульфату натрия хлорида железа (\RomanNumeralCaps{3}) цвет раствора сначала меняется на кроваво-красный(образуются комплексы), а затем окраска пропадает, итоговое уравнение:
\begin{equation}
    2FeCl_3 + Na_2S_2O_3 + H_2O \xrightarrow[]{} 2FeCL_2 + Na_2SO_4 + 2HCl
\end{equation}
\subsection{Опыт № 11. Качественное обнаружение соединений азота.}
Ионы аммония обнаружить довольно просто, мы сделаем это основываясь на основных свойствах аммиака:
\begin{equation}
    NH_4Cl + NaOH \xrightarrow[]{} NH_3\uparrow + H_2O + NaCl
\end{equation}
Выделяющийся аммиак окрасит бумажку, смоченную универсальным индикатором в синий цвет.\newline
Нитрат ионы мы обнаружим благодаря их оксилительным свойствам:
\begin{equation}
    NaNO_3 + 4Zn + 7NaOH + 6H_2O \xrightarrow[]{} NH_3\uparrow +4K_2[Zn(OH)_4]
\end{equation}
Выделяющийся аммиак как обычно окрасит универсальный индикатор в синий. \newline
Далее нам предстоит определить нитрит ионы. Для начала мы посмотрим на диспропорционирование азотистой кислоты (подкисленный раствор нитрита):
\begin{equation}
    3HNO_2 \xrightarrow[]{} HNO_3 + 2NO + H_2O
\end{equation}
Пробирка окрашивается в голубоватый цвет, а над раствором появляется бесцветная (NO) и затем бурая прослойка ($NO_2$).\newline
Далее мы определим нитрит по его восстановительным свойствам и хорошей растворимости нитрата бария(в отличие от сульфата бария):
\begin{equation}
    5NaNO_2 + 2KMnO_4 + 6HCl \xrightarrow[]{} 5NaNO_3 + 2KCl + 2MnCl_2 + 3H_2O
\end{equation}
\begin{equation}
    NaNO_3 + BaCl_2 \xrightarrow[]{} \text{реакция не идет}
\end{equation}
Далее нам предстоит осуществить реакцию бурого кольца. Последовательно опишем реакции, происходящие в растворе. Первым делом сульфат железа в присутствии концентрированной серной восстанавливает азотную кислоту до NO:
\begin{equation}
    2NaNO_3 + 6FeSO_4 + 4H_2S0_4 \xrightarrow[]{} 2NO  + Na_2SO_4 + 3Fe_2 (SO_4)_3 + 4H_2O,
\end{equation}
Затем NO образует комплекс (бурое кольцо) с сульфатом железа 2:
\begin{equation}
    NO + [Fe(H_2O)_6]SO_4 \xrightarrow[]{} [Fe(H_2O)_5NO]SO_4
\end{equation}
\subsection{Опыт №12. Качественные обнаружение соединений фосфора}
Обнаружить фосфат ионы можно с помощью РИО с солями серебра, фосфат серебра - желтоватый осадок:
\begin{equation}
    3AgNO_3 + Na_3PO_4 \xrightarrow[]{} Ag_3PO_4\downarrow + 3NaNO_3
\end{equation}
А пирофосфатов у нас, к сожалению, не было.
\begin{figure}[H]
    \centering
    \includegraphics[width= 0.4\linewidth]{IMAGE 2024-04-04 23:56:43.jpg}
    \caption{Зато у нас был андрей хохлов и таракан}
    \label{fig:enter-label}
\end{figure}
\subsection{Опыт №13. Некоторые свойства фосфора.}
При нагревании красного фосфора образуются тяжелые пары белого фосфора, которые затем оседали во второй колбе.
\end{document}
