\documentclass[a4paper,12pt]{article} 
\usepackage{geometry}
\geometry{
	a4paper,
	total={170mm,257mm},
	left=20mm,
	top=20mm,
}
\usepackage{titlesec}
\titlelabel{\thetitle.\quad} %точка в section

%%% Работа с русским языком
\usepackage{cmap}                           % поиск в PDF
\usepackage{mathtext} 			 	       % русские буквы в формулах
\usepackage[T2A]{fontenc}               % кодировка
\usepackage[utf8]{inputenc}              % кодировка исходного текста
\usepackage[english,russian]{babel}  % локализация и переносы

%Математика
\usepackage{amsmath,amsfonts,amssymb,amsthm,mathtools} % AMS
\usepackage{icomma} % "Умная" запятая

%% Шрифты
\usepackage{euscript}	 % Шрифт Евклид
\usepackage{mathrsfs} % Красивый матшрифт

%% Команды
\DeclareMathOperator{\const}{\mathop{const}}

%% Перенос знаков в формулах
%\newcommand*{\hm}[1]{#1\nobreak\discretionary{}
%	{\hbox{$\mathsurround=0pt #1$}}{}}
\usepackage[pdftex]{graphicx}

%%% Заголовок
\author{Хохлов Андрей, Коротков Антон}
\title{Практическая работа 6 \\
	\textbf{Химические свойства галогенов и их соединений}}
\date{\today}

\begin{document}
	
	{\Large \maketitle}
\section{Взаимодействие щелочных металлов с водой}
Натрий хранят в керосине, ведь щелочной металл активно реагирует с кислородом, после того как мы вытащили и порезали кусок, срез явного металлического цвета, но со времением он начинает мутнеть. Металл очень мягкий.
  
\begin{equation} 
\mathrm{2Na + O_2 \longrightarrow Na_2O_2  } 
\end{equation}

Капнули в чашу с водой фенолфталеина и положили кусок натрия на воду. Он начал двигаться по поверхности и бурно шипеть, если бы концентрация была больше, то произошёл бы взрыв.

\begin{equation} 
\mathrm{Na + H_2O \longrightarrow NaOH + H_2 } 
\end{equation}

Если провести аналогичный опыт с литием,  то вести он себя будет примерно так же, реагировать менее бурно. Интересный факт - литий почти никогда не взрывается в воде, ведь атомный радиус лития меньше, чем у натрия, а активность растёт вниз по группе у щелочных металлов.
\begin{equation} 
\mathrm{2Li + 2H_2O \longrightarrow 2LiOH + H_2 } 
\end{equation}
\section{Взаимодействие кальция с водой}
Подготовили прибор, состоящий из лабораторного штатива и расположенного на основании
штатива кристаллизатора, заполненного водой. В лапке штатива закрепили вертикально пробирку
отверстием вниз. Пробирку вынули из штатива и заполнили водой
доверху, закрыли ее отверстие резиновой пробкой(степашкиной рукой), погрузите ее в кристаллизатор. Затем
подняли ее вверх дном из воды так, чтобы отверстие пробирки оставалось погруженным в воду,
но находилось на расстоянии от дна более 1 см. Закрепили пробирку в лапке штатива в исходном
положении. 
Потом взяли из банки небольшое количество кальциевой стружки при помощи
микрошпателя и поместили их на лист сухой фильтровальной бумаги. Завернули кальциевую
стружку в кусочек марли, взяли ее пинцетом и внесли в кристаллизатор с водой, расположив
кальций сразу под отверстием пробирки.  Следите за тем, чтобы выделяющийся водород собирался в пробирку, вытесняя из нее воду.


\begin{equation} 
\mathrm{Ca + 2H_2O \longrightarrow Ca(OH)_2 + H_2 } 
\end{equation}
Белые хлопья,
образующиеся в кристаллизаторе это гидроксид кальция(II), он малорастворим поэтому существует в виде взвеси. После того, как весь кальций прореагирует, а водород полностью заместит воду в пробирке, мы взяли пробирку и подожгли водород. Произошёл \textbf{ХЛОПОК ГАЗА}

\begin{equation} 
\mathrm{2H_2 + O_2 \longrightarrow 2H_2O } 
\end{equation}
\section{Cвойства раствора гидроксида кальция}
В первой пробирке - хлорид натрия
\begin{equation} 
\mathrm{Ca(OH)_2 + NaCl \longrightarrow \times } 
\end{equation}
Во второй пробирке - хлорид аммония
\begin{equation} 
\mathrm{Ca(OH)_2 + 2NH_4Cl \longrightarrow CaCl_2 + 2NH_3 + 2H_2O } 
\end{equation}
В третьей пробирке - фенолфталеин, даёт цвет на \textbf{щелочную среду}


В четвёртой пробирке сульфат натрия, даёт белый осадок
\begin{equation} 
\mathrm{Ca(OH)_2 + Na_2SO_4 \longrightarrow CaSO_4 + 2NaOH } 
\end{equation}
\section{Окраска пламени солями щелочных и щелочноземельных металлов}
Взяли петлю в виде ушка из стальной проволоки, промыли ее в соляной кислоте и
прокалили в пламени портативной газовой горелки (в вытяжном шкафу). Взяли бюксы с
насыпанными в них солями щелочных и щелочноземельных металлов. По очереди опустили
петлю в каждый бюкс и внесли в пламя горелки.
\begin{table}[!ht]
    \centering
    \begin{tabular}{|l|l|l|l}
    \hline
        Металл & Энергия возбуждения, эВ & Длина волны излучения, нм &  Цвет пламени\\ \hline
        Литий & 1.9 & 670.8&   Малиновый \\ \hline
        Натрий & 2.1 & 589.0&  Жёлтый \\ \hline
        Калий & 1.6 & 766.5& Фиолетовый\\ \hline
        Кальций & 2.9–3.2 & 422.7& Кирпично-красный \\ \hline
        Стронций & 2.7–3.0 & 460.7& Красный \\ \hline
        Барий & 2.2–3.8 & 553.5& Зелёный \\ \hline
    \end{tabular}
\end{table}
\section{Взаимодействие магния с водой}
Без нагревания не идёт реакция, при нагреве идёт.
\begin{equation} 
\mathrm{Mg + H_2O \longrightarrow Mg(OH)_2 + H_2} 
\end{equation}
При добавлении фенолфталеина даёт щелочную реакцию(фиолетовую окраску)
При добавлении хлорида аммония летит газ
\begin{equation} 
\mathrm{Mg + NH_3Cl \longrightarrow MgCl_2 + NH_3 + H_2O} 
\end{equation}
\section{Взаимодействие алюминия с разбавленными растворами кислот}
Реакции идут только при НАГРЕВЕ
\begin{equation} 
\mathrm{2Al + 6HCl \longrightarrow 2AlCl_3 + 3H_2} 
\end{equation}
\begin{equation} 
\mathrm{2Al + 3H_2SO_4 \longrightarrow Al_2(SO_4)_3 + 3H_2} 
\end{equation}
\begin{equation} 
\mathrm{2Al + 6HNO_3 \longrightarrow 2Al(NO_3)_3 + 3H_2} 
\end{equation}
\section{Взаимодействие алюминия с раствором щелочи}
При помещении в раствор щёлочи без нагрева, алюминий покрыт оксидной плёнкой, при нагревании она разваливается и реакция идёт.
\begin{equation} 
\mathrm{Al + NaOH + H_2O \longrightarrow Na[Al(OH)_4] + H_2}
\end{equation}
\section{Пассивация алюминия}
Концентрированная азотная кислота убирает оксидную плёнку и позволяет алюминию бурно реагировать с соляной кислотой
\begin{equation} 
\mathrm{2Al + 6HCl \longrightarrow 2AlCl_3 + 3H_2} 
\end{equation}
\section{Активация алюминия}
При добавлении алюминия в раствор сульфата меди идёт следующая реакция:
\begin{equation} 
\mathrm{2Al + CuSO_4 \longrightarrow Cu + Al_2(SO_4)_3} 
\end{equation}
При этом медь проявляется спустя долгое время, небольшими пятнышками, ведь из-за оксидной плёнки алюминий не может полностью реагировать с медью из раствора.


Добавляя хлорид ион, он помогает убрать оксидную плёнку и ускоряет реакцию, медь проявляется быстрее на большей площади.

Убрав оксидную плёнку кислотой, алюминий почти мгновенно покрылся медной плёнкой.
\section{Получение гидроксида алюминия и изучение его свойств}
\begin{equation} 
\mathrm{AlCl_3 + 3NH_3 + 3H_2O \longrightarrow Al(OH)_3 + 3NH_4Cl} 
\end{equation}
Потом, полученный осадок гидроксида разделили на 2 пробирки
В первой пробирке:
\begin{equation} 
\mathrm{Al(OH)_3 + HCl \longrightarrow AlCl_3 + H_2O} 
\end{equation}
Во второй пробирке:
\begin{equation} 
\mathrm{Al(OH)_3 + NaOH \longrightarrow Na[Al(OH)_4]} 
\end{equation}


\end{document}
